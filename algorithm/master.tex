\documentclass{amsart}
\usepackage{tikz}

\gdef\R{\rightarrow}

\newtheorem{lemma}[subsection]{Lemma}

\begin{document}

\null
\vskip 3cm plus 1cm
\title{Syntax tree transformation as dialgebraic fold}
\maketitle
\vskip 3cm plus 1cm

\section{Syntax tree transformations by example}

Let us see some examples of transformations between grammars.

\subsection{Concrete and abstract grammars for arithmetic expressions}

The input grammar describes the concrete syntax of arithmetic
expressions with operator precedence and parentheses.

Let $C$ be the concrete grammar.
\begin{align*}
C &\R C~+~S & (C_1) \\
C &\R S     & (C_2) \\
S &\R S~\times~F & (S_3) \\
S &\R F     & (S_4) \\
F &\R \mbox{integer} & (F_5) \\
F &\R (~C~) & (F_6)
\end{align*}

Let $A$ be the abstract grammar.
\begin{align*}
A &\R A~+~A & (A_1) \\
A &\R A~\times~A & (A_3) \\
A &\R \mbox{integer} & (A_5)
\end{align*}

Intuitively, nodes labeled $C_1,S_3,F_5$ in concrete syntax trees
gets translated respectively to nodes labeled $A_1,A_3,A_5$ in
abstract syntax trees.

This is the concrete syntax tree of the expression $3\times4+5$.
\[
\begin{tikzpicture}
\path node{$C_1$}
child { node{$C_2$}
  child { node{$S_3$}
    child { node{$S_4$} child { node{$F_5$} child { node{$3$} }}}
    child { node{$\times$} }
    child { node{$F_5$} child { node{$4$} } }
  }
}
child { node{$+$} }
child{ node{$S_4$} child{ node{$F_5$} child{ node{$5$} }}}
;
\end{tikzpicture}
\]

This is the abstract syntax tree of the expression $3\times4+5$.
\[
\begin{tikzpicture}
\path node{$A_1$}
child { node{$A_3$}
  child { node{$A_5$} child { node{$3$} } }
  child { node{$\times$} }
  child { node{$A_5$} child { node{$4$} } }
}
child { node{$+$} }
child { node{$A_5$} child { node{$5$} } }
;
\end{tikzpicture}
\]

\subsection{Chomsky normal form} The input grammar is the
abstract grammar $A$ of arithmetic expressions. The output
grammar $N$ is the Chomsky normal form of $A$.
\begin{align*}
N &\R N~P & (N_1) \\
N &\R N~Q & (N_3) \\
N &\R \mbox{integer} & (N_5) \\
P &\R +~N & (P_1) \\
Q &\R \times~N & (Q_3)
\end{align*}

Intuitively, nodes labeled $A_1,A_3,A_5$ in abstract syntax trees
gets translated respectively to nodes labeled $N_1,N_3,N_5$ in
syntax trees in Chomsky normal form. The production $P_1$ comes
from $A_1$, and the production $Q_3$ comes from $A_3$.

This is the syntax tree of $3\times4+5$ in Chomsky normal form.
\[
\begin{tikzpicture}
\path node{$N_1$}
child { node{$N_3$}
  child { node{$N_5$} child { node{$4$} } }
  %child[draw opacity=0] {}
  child { node{$Q_3$}
    child { node{$\times$} }
    child { node{$N_5$} child { node{$5$} } }
  }
}
child[draw opacity=0] {}
child { node{$P_1$}
  child { node{$+$} }
  child { node{$N_5$} child { node{$2$} } }
};
\end{tikzpicture}
\]

\section{Information flows by example}

Each transformation between two grammars gives rise to a syntax
tree transformation. We will describe syntax tree transformations
by the information flow in the grammar transformation.

\subsection{Concrete and abstract grammars for arithmetic expressions}

To describe information flows, we need a unique identifier for
all terminals and nonterminals on the right-hand-side of a
production rule. Let us index nonterminals by superscripts.

This is the result of indexing the concrete grammar $C$.
\begin{align*}
C &\R C^1~+^2~S^3 & (C_1) \\
C &\R S^4     & (C_2) \\
S &\R S^5~\times^6~F^7 & (S_3) \\
S &\R F^8     & (S_4) \\
F &\R \mbox{integer}^9 & (F_5) \\
F &\R (~C^{10}~) & (F_6)
\end{align*}

This is the result of indexing the abstract grammar $A$. We add
a special trivial equation $A_0$, which corresponds to the
productions $C_2,S_4,F_6$ that are removed in the abstract
grammar. $A_0$ is not a real production; there are no $A_0$ nodes
in abstract syntax trees.
\begin{align*}
A &= A^0       & (A_0) \\
A &\R A^2~+^3~A^4 & (A_1) \\
A &\R A^5~\times^7~A^8 & (A_3) \\
A &\R \mbox{integer}^{10} & (A_5)
\end{align*}

Once nonterminals have unique identifiers, we can talk about
information flow. Let us describe the transformation from
concrete syntax trees to abstract syntax trees.
\begin{itemize}
\item $C_1$ nodes transform into $A_1$ nodes with information
flowing from $C^1$ to $A^2$, from $+^2$ to $+^3$, and from $S^3$
to $A^4$.
\item $C_2$ nodes transform into unknown abstract nodes with
information flowing from $S^4$ to $A^0$.
\item $S_3$ nodes transform into $A_3$ nodes with information
flowing from $S^5$ to $A^5$, from $\times^6$ to $\times^7$, and
from $F^7$ to $A^8$.
\item $S_4$ nodes transform into unknown abstract nodes with
information flowing from $F^8$ to $A^0$.
\item $F_5$ nodes transform into $A_5$ nodes with information
flowing from $\mbox{integer}^9$ to $\mbox{integer}^{10}$.
\item $F_6$ nodes transform into unknown abstract nodes with
information flowing from $C^{10}$ to $A^0$.
\end{itemize}

\subsection{Chomsky normal form}

This is the result of indexing the Chomsky normal form $N$ of the
abstract grammar $A$.
\begin{align*}
N &\R N^1~P^2 & (N_1) \\
N &\R N^3~Q^4 & (N_3) \\
N &\R \mbox{integer}^5 & (N_5) \\
P &\R +^6~N^7 & (P_1) \\
Q &\R \times^8~N^9 & (Q_3)
\end{align*}

Let us describe the transformation from syntax trees of $A$ to
syntax trees of $N$.
\begin{itemize}
\item $A_1$ nodes transform into $N_1$ nodes with information
flowing from $A^2$ to $N^1$, from $+^3$ through $P^2$ to $+^6$,
and from $A^4$ through $P^2$ to $N^7$.
\item $A_3$ nodes transform into $N_3$ nodes with information
flowing from $A^5$ to $N^3$, from $\times^7$ through $Q^4$ to
$\times^8$, and from $A^8$ through $Q^4$ to $N^9$.
\item $A_5$ nodes transform into $N_5$ nodes with information
flowing from $\mbox{integer}^{10}$ to $\mbox{integer}^5$.
\end{itemize}

\section{Information flows, formally}

An information flow between two context-free grammars can be
considered a collection of rewrite rules. We will define it
precisely here.

An \textbf{indexed grammar} is a context-free grammar with the
following changes:
\begin{enumerate}
\item Every occurrence of every symbol on right-hand-side of
production rules receives a unique, identifying superscript.
\item For each nonterminal $A$ in the grammar, add the equation
\[
A=A^0\qquad(A_0).
\]
\end{enumerate}

From now on, we assume the existence of two indexed grammars, an
\textbf{input} grammar and an \textbf{output} grammar.

An \textbf{information path} is a sequence of symbols
\[
(S^m,\ldots,S^2,S^1,T^1,T^2,\ldots,T^n),
\]
such that
\begin{enumerate}
\item $S^m,\ldots,S^1$ are terminal or nonterminal symbols of the
input grammar,
\item $T^1,\ldots,T^n$ are terminal or nonterminal symbols of the
output grammar,
\end{enumerate}
For example, in the transformation from the abstract grammar $A$
to the grammar $N$ in Chomsky normal form, the information path
from $+^3$ to $+^6$ is the sequence $(+^3,P^2,+^6)$.

An \textbf{information flow} is a collection of flow fragments. A
\textbf{flow fragment} is basically a rewrite rule between
different languages. Formally, a flow fragment is a tuple
$(Q,R,C)$, where
\begin{enumerate}
\item $Q$ is a production rule or equation in the input grammar,
\item $R$ is a production rule or equation in the output grammar,
\item $C$ is a collection of information paths.
\end{enumerate}

This is the information flow from the concrete grammar $C$ to the
abstract grammar $A$:
\begin{align*}
(C_1,A_1,&\{(C^1,A^1),(+^2,+^3),(S^3,A^4)\})\\
(C_2,A_0,&\{(S^4,A^0)\})\\
(S_3,A_3,&\{(S^5,A^5),(\times^6,\times^7),(F^7,A^8)\})\\
(S_4,A_0,&\{(F^8,A^0)\})\\
(F_5,A_5,&\{(\mbox{integer}^9,\mbox{integer}^{10})\})\\
(F_6,A_0,&\{(C^{10},A^0)\})
\end{align*}

This is the information flow from the abstract grammar $A$ to the
grammar $N$ in Chomsky normal form:
\begin{align*}
(A_1,N_1,&\{(A^2,N^1),(+^3,P^2,+^6),(A^4,P^2,N^7)\})\\
(A_3,N_3,&\{(A^5,N^3),(\times^7,Q^4,\times^8),(A^8,Q^4,N^9)\})\\
(A_5,N_5,&\{(\mbox{integer}^{10},\mbox{integer}^5)\})
\end{align*}


\section{Well-formed and well-behaved information flows}

This section describes the form an information flow must take in
order to correspond to a syntax tree transformation. Intuitively,
a well-behaved information flow describes a pattern-matching
expression, a well-behaved flow fragment describes one case in
the pattern-matching expression, and a well-behaved information
path describes one variable in one case of the pattern-matching
expression. Since variables are described by paths, they have
exactly one binding occurrence and exactly one bound occurrence.
In other words, case expresions described by information flows
are necessarily linear.

\bigbreak

$(S^m,\ldots,S^1,T^1,\ldots,T^n)$ is a \textbf{well-behaved
information path} if
\begin{itemize}
\item for all $2\le i\le m$, the symbol of $S^{i-1}$ is the
left-hand-side of the production rule containing $S^i$, and
\item for all $n\ge j\ge 2$, the symbol of $T^{j-1}$ is the
left-hand-side of the production rule containing $T^j$.
\end{itemize}

\bigbreak

$P$ is a \textbf{compatible set} of information paths if for any
two information paths $(S^m,\ldots,S^1,T^1,\ldots,T^n)$ and
$(U^h,\ldots,U^1,V^1,\ldots,V^k)$ in $P$,
\begin{itemize}
\item there exists $i,j$ such that $S^i\neq U^i$ and $T^j\neq
V^j$,
\item if $S^i=U^i$ for all $i<x$, then $S^x$ and $U^x$ occurs in
the same production,
\item if $T^j=V^j$ for all $j<y$, then $T^y$ and $V^y$ occurs in
the same production.
\end{itemize}

\bigbreak

$(Q,R,C)$ is a \textbf{well-behaved flow fragment} if
\begin{itemize}
\item $C$ is a compatible set of well-behaved information paths,
\item for each $(S^m,\ldots,S^1,T^1,\ldots,T^n)$ in $C$,
\begin{itemize}
\item $S^1$ occurs on the right-hand-side of the production $Q$,
\item $T^1$ occurs on the right-hand-side of the production $R$.
\end{itemize}
\end{itemize}

\bigbreak

A \textbf{well-behaved information flow} consists of well-behaved
flow fragments. The \textbf{inverse} of an information flow is
obtained by reversing each information path and swapping the
input and output symbols of each flow fragment.

\begin{lemma}
The inverse of a well-behaved information flow is well-behaved.
\end{lemma}



\section{Tree patterns and pattern-matching}

Pattern matching seems easier to describe than information flows.
Consider describing syntax tree transformations by case
expressions instead of information flows.



\section{Information flows as dialgebras}

From a well-formed and well-behaved information flow, we generate
a syntax tree transformation in 3 steps.
\begin{enumerate}
\item View the input grammar as the fixed point $\mu F$ of an
$n$-nary higher-order functor $F$. View the output grammar as the
fixed point $\mu G$ of an $n$-nary higher-order functor $G$.
\item Compile the information flow into an $F,G$-dialgebra of
type $F~\alpha\R G~\alpha$, where $\alpha$ can be either $\mu F$
or $\mu G$. Instantiating $\alpha$ to $\mu G$ produces the
$F$-algebra
\[
f : F~(\mu G) \R \mu G.
\]
Instantiating $\alpha$ to $\mu F$ produces the $G$-coalgebra
\[
g : \mu F \R G~(\mu F).
\]
\item The desired syntax tree transformation is given by the
catamorphism with respect to the $F$-algebra $f$. It is at the
same time the anamorphism with respect to the $G$-coalgebra $g$.
\end{enumerate}

\end{document}
